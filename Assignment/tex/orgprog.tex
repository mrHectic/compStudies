\chapter*{3 Organizational conversation}
\addcontentsline{toc}{chapter}{3 Organizational conversation}
\label{orgprog}
Once a preliminary understanding of the topic at hand (as alluded to in \textbf{\nameref{intro}} and expanded upon in \textbf{\nameref{lexp}}) has been gained, the organizations that are comprised of individuals like Mr Swiegers can be conversed with.
\\\\
Moxico Resources PLC is a private operating, development and exploration mining company and is part of the construction and engineering industry. It was chosen as the quintessential organization in the pertaining industry under investigation. Moxico Resources is involved with infrastructure development and project management as well as social development where they have recognised the importance of mentoring younger engineers and the effect this has on the economy. This is indeed a fact mentioned by the individual as per \textbf{\nameref{lexp}}.
\\\\
Moxico Resources has seen both successful and less successful projects and has come to the conclusion that the greater the amount of data and facts available, the greater the chance of project success. Therefore, Moxico has implemented a detailed planning and fact -finding intervention at their organisation. They have a dedicated team with the sole purpose of gathering data and facts about the site before the execution of the project.
\\\\
Social development includes the hiring of new engineers as well as the training and mentoring of younger engineers. Moxico believes that first impressions are not always the most accurate way of hiring people and will make hiring decisions based on data/facts about the individual. They are interested in the character of the person and look out for previously disadvantaged individuals. Moxico does not yet have a specific program to targets social development, but they are aware of its importance and plan to implement a strategy soon.
\\\\
It is evident that the more facts/data available before a project commences, the greater the chances that the project will succeed. From Mr Swiegers’ interview (in \textbf{\nameref{lexp}} it is evident that the costs of incomplete and inaccurate data are indeed extensive. Furthermore for \textbf{\nameref{lexp}}, it could be noted that most engineering or construction firms work according to the same principle: to complete a project within its schedule and for it to be successful. This is relatable, however, it's easy to forget about the bigger picture which is the development of South Africa in the context described earlier. Mr Swiegers mentioned that a professional engineer would not spend time to mentor a junior engineer, because it is a waste of time. This is probably because of a tight schedule for projects and professional engineers want to focus on finishing a project. This can impact junior engineers negatively in the future, because they do not get the right mentorship for when they themselves will have to work as professional engineers. This links to the first article discussed in \textbf{\nameref{lit}} where it was described how the lack of education is one of the causes of poverty and poor development in South Africa. 
\\\\
If younger engineers aren't getting enough experience or mentorship, it will have a negative impact on future projects and development in South Africa. Senior engineers should take time and make an effort to train younger engineers. This should greatly improve all spheres of development in South Africa as mistakes need not be repeated. 
