\chapter*{1 Introduction}
\addcontentsline{toc}{chapter}{1 Introduction}
\label{intro}
This document explores the various ways in which empirical knowledge (\textbf{facts} \& \textbf{data}) contribute towards the development of South Africa. It is apparent that such a broad and encompassing topic can not be conceptualized, and subsequently discussed, without focusing on specific areas of development. To facilitate useful conversation in this regard, this document will narrow its scope to three main areas of development. The three areas of focus are, the development of infrastructure, social development and the development (and therefore growth) of the economy.
\\\\
It is of particular importance for the reader to note that such a narrowing of scope is indeed justified. Development in one context will, indeed, lead towards development in another. 
\\\\
Firstly, a bottom-up approach is taken in order to investigate the aforementioned topic by assessing the \textbf{first hand account (or lived experience)} of engineers - the real world people involved in industry. This is discussed in \textbf{\nameref{lexp}}. Thereafter, it is important to address the organization that these first hand accounts belong to. This is done in \textbf{\nameref{orgprog}} The sum is indeed greater than the parts in this case, and a nuanced understanding can only be achieved by being in \textbf{conversation with organizations} in South Africa. Lastly, it is necessary to review the \textbf{literature} pertaining to the topic at hand in \textbf{\nameref{lit}}. 
\\\\
Once the bottom-up analysis has been completed as mentioned, a conclusion can be drawn from this. This is summarized in \textbf{\nameref{conc}}. In \textbf{\nameref{A}}, a questionnaire can be found. This questionnaire served as a guide for all conversation and dialogue referenced in this document.


%This report explores how facts/data contribute towards the development of South Africa. The way in which this topic was interpreted was firstly though classifying the various sectors of development. The sectors of developments we decided to focus on included infrastructure, social development and economic growth. Although keeping in mind that development in one sector often correlates with development in the others. Articles based on these sectors were researched, followed by identifying how statistics could be used to lead development in South Africa. From this research a questionnaire was formulated for an interview.
%\\\\ 
%When deciding on someone to interview, we looked for a person working in the engineering Industry that will have experience on how engineers operates. The person we settled on was a Project Manager, Renier Swiegers. He has the firsthand experience in the Engineering/construction industry, giving not only an engineer's overview but also details on decision making around Infrastructure.
