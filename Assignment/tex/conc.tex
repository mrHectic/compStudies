\chapter*{5 Conclusion}
\addcontentsline{toc}{chapter}{5 Conclusion}
\label{conc}
As assessed in this document, the inclusion of data is indeed imperative when it comes to the development of South Africa. Data, and the facts that emerge from datasets, serve as a substrate for development in South Africa. This is not only true in a broad sense, but also in the specific context discussed in \textbf{\nameref{lexp}}, \textbf{\nameref{orgprog}} and \textbf{\nameref{lit}}.
\\\\
In \textbf{\nameref{lexp}}, a first-hand perspective was gained through dialogue with a real-world person intimately involved in the construction and engineering industry. The pertaining individual, Mr Swiegers, highlighted the importance of accurate and factual data when it comes to infrastructure in South Africa. He alluded to how these same facts and datasets can indeed, lead towards the secondary effect of social development. He in fact, further showed how young engineers require the imparting of facts and knowledge by senior mentors in the industry. All dialogue with Mr Swiegers was dictated by the list of questions in \textbf{\nameref{A}}
\\\\
The conversation with Mr Swiegers provided a good background when contacting organizations involved in engineering in South Africa as per \textbf{\nameref{orgprog}}. Moxico Resources PLC, was contacted and interviewed as per \textbf{\nameref{A}}. The information garnered form the conversation with Moxico Resources PLC served to confirm the finding of \textbf{\nameref{lexp}}, but also shed light on the economic implications of using accurate data and factual observations. It was concluded that accurate project data is a precursor to project success. 
\\\\
Finally, \textbf{\nameref{lit}} serves as a top-down analysis of how engineers in South Africa use data and facts for development. This top-down analysis is reconcilable with the bottom-up approach taken in \textbf{\nameref{lexp}} and \textbf{\nameref{orgprog}}.