\chapter*{2 Lived Experiences}
\addcontentsline{toc}{chapter}{2 Lived Experiences}
\label{lexp}
We interviewed Mr Renier A. Swiegers, an experienced project manager for his opinion on how engineers can use data/facts in guiding/leading the development in South Africa. Mr Swiegers has several years of experience in dealing with engineers of varying experience levels as well as ensuring projects run smoothly with the limited data he initially receives.
\\\\ 
Mr Swieger's initial stance on the matter was based (very much) on the infrastructure development of South Africa. He says that engineers’ first resource is the national census as it allows engineers to appropriately plan for future development. He also stressed the importance of engineers knowing the facts before pursuing a project as poor planning due to lack of data could be the difference between the success and failure of a project. On a particular project that Mr Swiegers was managing in Zambia, there was a travel time of 3 months from South Africa. He said that if all the data/facts were not known to them, the entire project could be set back by several months. The possibility of fines, in addition to excessive courier costs also exists if important data is omitted.
\\\\
Mr Swiegers, furthermore, has seen more senior engineers facilitate the development of junior engineers through mentor-apprentice relationships. He moreover noted that not all professional engineers were willing to pass knowledge to others. Some engineers are very goal focused and find have do not prioritize imparting knowledge on others.
\\\\
When deciding upon new partnerships or hiring new employees, the first thing Mr Swiegers looks for is past experience. References form previous employers are indeed very important as they highlight the attitude of prospective employees. He says first impressions aren’t always the most accurate tool in judging a person’s character, but definitely contribute significantly to deciding whether to take them on or not.
\\\\
From experience, Mr Swiegers noticed that communities and people rely on engineers to provide them with safe, comfortable design that is easy to maintain. In addition to this, he believes that participating in training - both as student and teacher - is a fundamental means by which to contribute to the community.
From speaking with Mr Swiegers we could conclude that data and facts are definitely used in development. No matter which way you look at it; engineers will need to make use of data and facts to drive the economy, infrastructure and social development of South Africa.
\\\\
After the dialogue with Mr Swiegers, it is possible and necessary the move to a layer of abstraction above the individual. This is discussed in the next section: \textbf{\nameref{orgprog}}.
