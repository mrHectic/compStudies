\chapter*{4 Literature Review}
\addcontentsline{toc}{chapter}{4 Literature Review}
\label{lit}
As the final part of the step-wise analysis detailed in \textbf{\nameref{intro}}, a review of written work on the topic at hand is required. The two articles below were chosen as reputable sources and are therefore assessed.
\section*{4.1 Article A: Overcoming Poverty and Inequality in South Africa: An assessment of drivers, constraints and opportunities}
\addcontentsline{toc}{section}{4.1 Article A: Overcoming Poverty and Inequality in South Africa: An assessment of drivers, constraints and opportunities}
A synopsis of the significant ideas regarding poverty and inequality in South Africa is presented henceforth. It is noted that there are higher levels of poverty in rural areas than in urban areas. An assessment of drivers, constraints and opportunities are made known in the literature. The level of inequality and poverty and the people affected is also emphasized. \cite{WorldBankGroup2018}
\\\\
The poverty level worsened between 2011 and 2015 and it is recognized that the highest poverty rates are consistently exposed as Black South Africans. The unemployment rate has increased drastically since 2008 whilst obtaining a post-secondary qualification is no longer a safeguard against joblessness. Moreover, the youth unemployment rate remains awfully high at about 40\%. Gender inequality is also prevalent as there is a higher probability of males getting skilled employment as opposed to females. \cite{WorldBankGroup2018}
\\\\
There are, however, strategies that can be used to combat these issues. It is observed that with rising levels of education, poverty tends to decline. Utilizing this strong correlation, education should be promoted to reduce the absurd poverty levels in South Africa. The unemployment rate is another infamous issue that South African’s face, especially among/amongst the youth. Providing opportunities for the youth is therefore crucial to alleviate this issue. Furthermore, outreach programs can be implemented to develop poor communities. Uplifting the youth and underprivileged people of South Africa is paramount to creating a better future and breaking the cycle of poverty. \cite{WorldBankGroup2018}


\section*{4.2 Article B: Infrastructure development: What, Why, How and Who?}
\addcontentsline{toc}{section}{4.2 Article B: Infrastructure development: What, Why, How and Who?}
In this article Thina Manga focuses on infrastructure development as a means to develop South Africa. The article focuses on the what, why, how and who aspects of infrastructure development as the title indicates. \cite{Manga2019}
\\\\
Investing in the country’s infrastructure is crucial now as the South African economy has fallen by 3.2\% between 2018 and 2019. Manga believes the way of fixing the weak economic environment that South Africa is in is by making use of funds such as the Futuregrowth Infrastructure \& Development Bond Fund. These funds are put in place to invest in infrastructural, social, environmental and economic development in South Africa \cite{Manga2019}.
\\\\
If South Africa were to make use of such funds the country could further develop infrastructure such as electricity sector. We could make use of data and facts as Manga does in the article to decide on a partner to invest in. Eliminating load shedding would have a positive impact on South Africa’s economy and thus its citizens. The only way of eliminating load shedding would be to make use of data and facts. Only once data and facts have been collected and analysed could we determine why not having electricity is an issue and how we could fix this issue. \cite{Manga2019}
\\\\
Manga concluded by stating that investing in infrastructure will have a positive impact on the overall economy of the country and thus will benefit all South Africans. \cite{Manga2019} 
